\section{Uczenie maszynowe (ang. Machine learning)}
    Uczenie maszynowe to technika pozwaljąca na naukę komputera pewnych wzorców występujących w danych. 
    Osiąga się to poprzez proces treningu, którego ostatecznym celem jest zminimalizowanie wartości funkcji kosztu 
    co oznacza, że celność modelu jest największa. Sam proces polega na iteracyjnych poprawkach - po każdej iteracji 
    powwiny być wprowadzone ulepszenia do modelu tak aby zmniejszyć funckję kosztu. Podczasz wykoniwania jednego obiegu 
    pętli model stara się odgadnąć co oznaczają dane - przewidywanie wartości, klasyfikowanie danych wejściowych na grupy.
    Na potawie wyjścia z modelu można obliczyć funckję kosztu co daje możliwosć określenia w którym kierunku model 
    powinien być modyfikowany tak aby zwiększyć jego skuteczność.  
    pozwala na rozwiązanie wielu problemów takich jak: przewydywanie, klasyfikacja, znajdowanie anomali w sekwencjach itd. \\ 
    Współćześnie odroznia się trzy gałęzie uczenia maszynowego:
    \begin{itemize}
        \item Unsupervised learning
        \item Suppervised learning
        \item Reinforcmet learning
    \end{itemize} 
