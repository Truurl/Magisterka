\chapter{Część praktyczna}
    \justify
\chapter{Eksperymenty i wyniki}
	\justify
	\section{Wybór metryk jakości obrazów}
        \subsection{FID (Fréchet inception distance)}
            Jednym z większych problemów modelów t
            ypu GAN, jest wybór metryki, która będzie dobrze oceniać czy procesz szkolenia przebiega pomyślnie. Porónywanie syntetycznych orbazów z relanymi wymaga numarycznego przestawienia czy zbiór cech generowanych obrazów zawiera się w zbierze cech obrazów realnych. Metryka FID opiera się an analizie statyyczc znej cech orabzów, które są wudobyane z przeszkolonego modelu \textit{InceptionV3}
        \subsection{Dywergencja Kullbacka-Leiblera}
            