
\chapter{Przegląd literatury}
    \justify    
    \section{Domain adaptation}
            W wielu przypadkach uczenie maszynowe zakłada, że zbióry danych treningowe i testowe (również rozumiane jako realne środowisko) pochodzą
        z identycznych rozkładów lub posiadają podobny rozkład cech. Przyjmując takie założenia można spodziewać się, że poprawnie wyszkolony model
        nie powinien tracić na jakości, gdy zmienni się jego środowisko pracy - przenosząc go ze środowisko szkoleniowego do ptawdziwego życia.
        Praktyka pokazuję, że takie założenie jest błędne, typowym jest to, że jakość działania modelu spada przy zmiannach środowiska pracy. 
        Spowodowane jest to, rożnicą w dystrybicji zbirów danych (testowych i treningowych) oraz realnym środowiskiem. Powodem różnych dystrybucji
        danych jest trudność w zdobywaniu danych o tych samych własnościach, żródeł itp. dodatkowo przywykło się do stotowania publicznych 
        zbiorów danych - tworzenie nowych jest bardzo kosztowe oraz czasochłonne\cite{domain_adaptation_review}.
        \par
            Zagadnienie \textit{domain adaptataion} związane jest z następującymi pojęciami. Zbiorry treningowe i testowe nazywane są jako 
        \textit{source domain} i \textit{target domain}. Różnicę w rozkładach w angielskiej numeroklaturze określa się zwrotem \textit{domain gap},
        natomiast pogorszenie jakości działania modelu nazywa się \textit{domain shift}. Podane zwroty będą stosowane naprzemmienie z polskimi 
        tłumaczeniami. 
        \par  
            Rozwiązaniem powyższego problemu jest nowa dziedzina związana z uczeniem mawszynowynm \textit{Domain adaptation}. Techniki adaptacji domeny
        pozwalają modelom uczenia mszynowego na odowiednią naukę taką, aby mogły sobie świetnie radzić w realnych zastotowaniach czy też w zbioprach 
        testowych. \textit{Doman adaptation} jest to w pewnym rozumiemu \textit{transfer learning}\cite{transfer_learning}. Obie dziedziny starają się
        rozwiązać ten sam problem ale na dwa różne sposoby. Wspólnym mianownikiem obu przypadków jest to, że starają się poprawić jakość działania modelu
        mając nie wystarczającą ilość danych (rozumianą jako \textit{target domain}) poprzez wykorzystanie wiedzy wydobytej z innego zbioru danych 
        (\textit{source domain}). Przypadki \textit{transfer learning} to sytuacje gdzie zadanie modelu jest zmieniane, ale domena może być pozostawiona lub może
        się zmienić pomiędzy \textit{source} i \textit{target}. Natomiast w sytuacjach \textit{domain adaptation} zadnaie dla modelu uczenia maszynowego się 
        nie zmienia, za to \textit{source domain} i \textit{target domain} rożnią się wzgledem siebie. \par
            Istnieją jeszze inne metody radzenia z wyżej opisanym problemem. \textit{Semi-supervised classification}\cite{semi-sup-learning} jest jedną z takich 
        metod. Polega ona, na tworzeniu modeli uczenia maszynowego, którą korzystają z danych z etykietami oraz bez etykiet (ang. label), przy czym zakłada się, 
        że oba zbiory danych pochodźa z tego samego rozkładu prawpodobieństwa. Kolejną metodą jest \textit{Multi task learning}\cite{multitask-learning}, które
        stara się polepszyć generalizację pomiędzy wioeloma spokrewinonymi zadaniami poprzez jednoczesne szkolenie. Można, załozyć że podbne zadania będą 
        musiały korzystać ze wspołnych informacji, Model uczony w taki sposób, zmuszony jest do dokładnego poznania struktury danych oraz współdzielenia 
        reprezentacji danych pomiędzy różnymi zadaniami. \textit{Multi task learning} i \textit{Transer learning} stosują podobne metody m.in wspóldzielenie 
        parametrów, transformacja cech, jednakże \textit{transer learning} jest wykorzystywany do poprawy działania modelu poprzez najpierw uczenia go na 
        \textit{source domain} a później transferu wydobytej wiedzy do \textit{target domain}. \textit{Multi task learning} skupia się na poprawy jakości 
        wielu spokrewinoch zadań poprzez wspólne uczenie.
         \pagebreak 
    \section{Domain gap}
        \pagebreak
    \section{Przegląd technik domain adaptation}
        \pagebreak
